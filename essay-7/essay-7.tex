\documentclass{article}
\usepackage[utf8]{inputenc}

\title{Essay 7 - Mount Tambora}
\author{Benny Chen}
\date{April 19, 2023}

\usepackage{color}
\usepackage{amsthm}
\usepackage{amssymb} 
\usepackage{amsmath}
\usepackage{listings}
\usepackage{xcolor}
\usepackage{listings}
\usepackage{graphicx}
\usepackage[hidelinks]{hyperref}

\definecolor{codegreen}{rgb}{0,0.6,0}
\definecolor{codegray}{rgb}{0.5,0.5,0.5}
\definecolor{codepurple}{rgb}{0.58,0,0.82}
\definecolor{backcolour}{rgb}{0.95,0.95,0.92}

\lstdefinestyle{mystyle}{
    backgroundcolor=\color{backcolour},   
    commentstyle=\color{codegreen},
    keywordstyle=\color{magenta},
    numberstyle=\tiny\color{codegray},
    stringstyle=\color{codepurple},
    basicstyle=\ttfamily\footnotesize,
    breakatwhitespace=false,         
    breaklines=true,                 
    captionpos=b,                    
    keepspaces=true,                 
    numbers=left,                    
    numbersep=5pt,                  
    showspaces=false,                
    showstringspaces=false,
    showtabs=false,                  
    tabsize=2
}

\lstset{style=mystyle}

\begin{document}

\maketitle

In this article the author talks about a our readiness for a mega eruption and the consequences that it would cause. The article talks about previous eruptions that happened like on Mount Tambora and Mount Rinjani. Another article that talks about this is from The Guardian. The Guardian made a article also talking about our readiness for a volcanic catastrophe and even gives more infomation about the past and consequences. The Guardian also agrees with many of this article's points, for example one fact that the article claims is ``In April 1815, the biggest known eruption of the historical period blew apart the Tambora volcano, on the Indonesian island of Sumbawa, 12,000km from the UK. What happened next testifies to the enormous reach of the biggest volcanic blasts''\footnote[1]{\url{https://rb.gy/6hpnv}}. I would give this article a 10/10. This is due to the article having very good factual statments and evidence. The article does very well in having cited sources. For example, one of the sources it has cited was a paper from GeoScienceWorld containing a forecast and predications of volcanic events throughout the world. There were many other articles and people cited like a article from Eos. I would give this article a 10/10 on having cited sources as all of them have links to these sources and whenever he uses them he states these sources. I did not find any instance of hyperboles in the article. The author presents the information in a factual and straightforward manner, without resorting to sensational or exaggerated language. The statements made are well-supported by data and scientific research, and the author encourages readers to approach the topic with a sober mindset. I would give the article a score of 9/10 for avoiding hyperbole. The author presents the information clearly and without excessive embellishment. The author presents nuance in several parts of the article, acknowledging the complex nature of studying and predicting volcanic eruptions. One example is when the author discusses the potential consequences of a VEI 7 eruption, distinguishing between proximal and distal zones of hazards and risk.The author acknowledges the limitations of our current understanding of volcanic activity, noting that while the recurrence frequency of VEI 7 eruptions is between one and two per thousand years, this does not guarantee complacency in preparing for future events. Overall, I would give the article a8/10 for presenting nuance. The author provides a good view of the topic, acknowledging both the potential consequences and the limitations of our understanding of volcanic activity.

\end{document}