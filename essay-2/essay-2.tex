\documentclass{article}
\usepackage[utf8]{inputenc}

\title{Essay 2 - Can we turn a desert into a forest?}
\author{Benny Chen}
\date{\today}

\usepackage{color}
\usepackage{amsthm}
\usepackage{amssymb} 
\usepackage{amsmath}
\usepackage{listings}
\usepackage{xcolor}
\usepackage{listings}
\usepackage{graphicx}
\usepackage[hidelinks]{hyperref}

\definecolor{codegreen}{rgb}{0,0.6,0}
\definecolor{codegray}{rgb}{0.5,0.5,0.5}
\definecolor{codepurple}{rgb}{0.58,0,0.82}
\definecolor{backcolour}{rgb}{0.95,0.95,0.92}

\lstdefinestyle{mystyle}{
    backgroundcolor=\color{backcolour},   
    commentstyle=\color{codegreen},
    keywordstyle=\color{magenta},
    numberstyle=\tiny\color{codegray},
    stringstyle=\color{codepurple},
    basicstyle=\ttfamily\footnotesize,
    breakatwhitespace=false,         
    breaklines=true,                 
    captionpos=b,                    
    keepspaces=true,                 
    numbers=left,                    
    numbersep=5pt,                  
    showspaces=false,                
    showstringspaces=false,
    showtabs=false,                  
    tabsize=2
}

\lstset{style=mystyle}

\begin{document}

\maketitle

In the article ``Can we turn a desert into a forest?'', the authors talk about a project that was done to create a vast green forest rather than a desert. After reading the article, I fact checked some of the material said to another article called, ``The Great Green Wall'' by National Geographic \footnote[1]{\url{https://education.nationalgeographic.org/resource/great-green-wall}}. In both articles, they claimed that the cause of the expanding desert is not just due to clmate change but also from deforestation, population growth, overgrazing, and much more. This claim seems to be correct in both articles and support each other. They also both support that fact of how much the area around the desert has imporved. Overall I rate this artle having a 7 in supporting its claims as most numerical claims has a orginaztion that supports it but there aren't any individual claims.

Cone cited source that the article used was how much land has the program restored. In the article it states how 990,000 acres were restored and cited the U.N. Convention to Combat Desertification. I think this article had a 9 when citing its sources but I didn't find any direct quotes from these sources. 

From what I can see, there is no use of a hyperbole in this article. I think this article has a 1 when it comes to using hyperbole. I did not find any exagerated statements in this article.

The article does have nuance, as it presents a complex issue of the Great Green Wall project and its impacts on different groups of people. The article highlights the successes of the program in terms of improving soil health and growing crops, but also acknowledges the challenges such as exclusion of certain groups, disruption of traditional land use agreements, and the need for alternative income-generating options. One quote that shows this is ``It's worth noting that too often land restoration efforts mainly benefit men like Mr. Alkali, who have access to large tracts and can take a gamble on a program that may seem mystifying or threatening to others.'' In this quote, the author shows that this porgram is good but not benefitial for everyone. I would give this article a 9 when it comes to nuance as it shows both sides of of this program.




\end{document}