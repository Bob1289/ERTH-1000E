\documentclass{article}
\usepackage[utf8]{inputenc}

\title{Essay 5 - Connecticut Dams}
\author{Benny Chen}
\date{\today}

\usepackage{color}
\usepackage{amsthm}
\usepackage{amssymb} 
\usepackage{amsmath}
\usepackage{listings}
\usepackage{xcolor}
\usepackage{listings}
\usepackage{graphicx}
\usepackage[hidelinks]{hyperref}

\definecolor{codegreen}{rgb}{0,0.6,0}
\definecolor{codegray}{rgb}{0.5,0.5,0.5}
\definecolor{codepurple}{rgb}{0.58,0,0.82}
\definecolor{backcolour}{rgb}{0.95,0.95,0.92}

\lstdefinestyle{mystyle}{
    backgroundcolor=\color{backcolour},   
    commentstyle=\color{codegreen},
    keywordstyle=\color{magenta},
    numberstyle=\tiny\color{codegray},
    stringstyle=\color{codepurple},
    basicstyle=\ttfamily\footnotesize,
    breakatwhitespace=false,         
    breaklines=true,                 
    captionpos=b,                    
    keepspaces=true,                 
    numbers=left,                    
    numbersep=5pt,                  
    showspaces=false,                
    showstringspaces=false,
    showtabs=false,                  
    tabsize=2
}

\lstset{style=mystyle}

\begin{document}

\maketitle

In this article, we read more about the history of dams in Connecticut and how demolishing them would affect wildlife. There are many statements made in this article and many have numerical values. One statement that I checked was that the number of dams in Connecticut is 4,000. I found this statement to be true. I found this statement to be true by looking at the Connecticut Department of Energy and Environmental Protection (DEEP) \footnote[1]{\url{https://portal.ct.gov/DEEP/Water/Dams/Introduction-to-Dams}}website along with other articles like the CTexaminer \footnote[2]{\url{https://ctexaminer.com/2019/09/30/old-dams-and-new-problems-for-connecticut-homeowners/}}. The DEEP website has a list of dams in Connecticut and the number of dams listed is 4,000. I feel like this article gets a 10 in supporting its claims as there are multiple sources to back them up. This article also has many quotes from these sources and from experts. One exmaple is from Sally Harold who talks about water levels and control. I also give this a 10 as there are many sources and citings for everything. From what I can see no, the author does not use hyperbole in this article. The article provides a straightforward and informative way of the efforts to remove dams in Connecticut to restore rivers and wildlife, without resorting to exaggeration. The author presents the facts in a clear manner and provides sources and facts. I would give this also a 10. In this article the author does also presents nuance. For example, the article mentions that while removing dams can have positive effects on the environment, it can also change the way a property looks and may be a tough sell for some property owners. Amy Singler from American Rivers states, "It's a lot to say 'Take our word for it,' because what we're saying is -- is it's going to look fundamentally different than what it looks like now." This shows that there are multiple perspectives to consider when it comes to removing dams, including the concerns of property owners. I would give this article a grade of 8 for presenting nuance, as it highlights some things involved in removing dams and acknowledges that there may be different perspectives to consider.




\end{document}