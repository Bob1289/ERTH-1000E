\documentclass{article}
\usepackage[utf8]{inputenc}

\title{Essay 1 - Poisoned Planet}
\author{Benny Chen}
\date{\today}

\usepackage{color}
\usepackage{amsthm}
\usepackage{amssymb} 
\usepackage{amsmath}
\usepackage{listings}
\usepackage{xcolor}
\usepackage{listings}
\usepackage{graphicx}
\usepackage[hidelinks]{hyperref}

\definecolor{codegreen}{rgb}{0,0.6,0}
\definecolor{codegray}{rgb}{0.5,0.5,0.5}
\definecolor{codepurple}{rgb}{0.58,0,0.82}
\definecolor{backcolour}{rgb}{0.95,0.95,0.92}

\lstdefinestyle{mystyle}{
    backgroundcolor=\color{backcolour},   
    commentstyle=\color{codegreen},
    keywordstyle=\color{magenta},
    numberstyle=\tiny\color{codegray},
    stringstyle=\color{codepurple},
    basicstyle=\ttfamily\footnotesize,
    breakatwhitespace=false,         
    breaklines=true,                 
    captionpos=b,                    
    keepspaces=true,                 
    numbers=left,                    
    numbersep=5pt,                  
    showspaces=false,                
    showstringspaces=false,
    showtabs=false,                  
    tabsize=2
}

\lstset{style=mystyle}

\begin{document}

\maketitle

In this article “Poisoned Planet”, Phil Plait talks about a catastrophe called the Great Oxygenation Event. The Great Oxygenation Event occurred around 2.5 billion years ago when Earth had nothing but some land and oceans. In these oceans, there existed some of the first bacteria. Life slowly came into form like Cyanobacteria. These Cyanobacteria produced oxygen as a waste from photosynthesis. As these Cyanobacteria kept producing oxygen, a problem slowly came where oxygen built up in the water, the water where other anaerobic bacteria existed, which became harmful for them as oxygen acted like a poison to them, causing the Great Oxygenation Event. In this context, the author described the Cyanobacteria as the polluters because the cause of this catastrophe is due to the overabundance of oxygen caused by the Cyanobacteria. They are polluters in this regard as they are filling the planet with something, in this case Oxygen. The analogy of it being pollution comes back to how the present day is and how we also do the same with burning fossil fuels to pollute the atmosphere. This however doesn’t work as the Cyanobacteria reason for pollution is different from our reason. Humans don’t have to constantly rely on this system for energy and can turn off any systems at any point while Cyanobacteria need to release oxygen as a waste product to survive. I do think the analogy of comparing the Cyanobacteria to humans is useful. I like this analogy since Cyanobacteria can create a big and negative change as a small organism while us humans numbering in the billions are doing the same at a fast rate. I like how the author shows how we should learn from the past to better the future. From the course book we can also read more about how humans actively think and know what we are doing and are seeking lower cost energy over the environment. Overall, this article shows us how we as humans are doing the same as the Cyanobacteria and polluting the earth.

\end{document}