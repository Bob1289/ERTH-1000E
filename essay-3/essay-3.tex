\documentclass{article}
\usepackage[utf8]{inputenc}

\title{Essay 3 - Sputtering Lithium sector threatens US electric vehicle goals}
\author{Benny Chen}
\date{\today}

\usepackage{color}
\usepackage{amsthm}
\usepackage{amssymb} 
\usepackage{amsmath}
\usepackage{listings}
\usepackage{xcolor}
\usepackage{listings}
\usepackage{graphicx}
\usepackage[hidelinks]{hyperref}

\definecolor{codegreen}{rgb}{0,0.6,0}
\definecolor{codegray}{rgb}{0.5,0.5,0.5}
\definecolor{codepurple}{rgb}{0.58,0,0.82}
\definecolor{backcolour}{rgb}{0.95,0.95,0.92}

\lstdefinestyle{mystyle}{
    backgroundcolor=\color{backcolour},   
    commentstyle=\color{codegreen},
    keywordstyle=\color{magenta},
    numberstyle=\tiny\color{codegray},
    stringstyle=\color{codepurple},
    basicstyle=\ttfamily\footnotesize,
    breakatwhitespace=false,         
    breaklines=true,                 
    captionpos=b,                    
    keepspaces=true,                 
    numbers=left,                    
    numbersep=5pt,                  
    showspaces=false,                
    showstringspaces=false,
    showtabs=false,                  
    tabsize=2
}

\lstset{style=mystyle}

\begin{document}

\maketitle

In the article ``Sputtering Lithium sector threatens US electric vehicle goals'', the authors talk about how the lithium sector, mainly in the US, is not doing well and how it is affecting the US electric vehicle goals. 

After reading the article, I fact checked some of the material said to another article called, ``How the US plans to transform its lithium supply chain'' by Utility Dive \footnote[1]{\url{utilitydive.com/news/us-strengthening-lithium-supply-processing-ev-batteries}}. In both articles, they claimed that the lithium sector is not doing well and that it is affecting the US electric vehicle goals. However, the article from Utility Dive states more about how the US is trying to strengthen its lithium supply chain. This claim seems to be correct in both articles and support each other. I fact checked that the US has only one lithium producer which is in Nevada. The article I found laso stated that the US only has 3.6\% of the worlds lithium reserves. I would give this a rating of 8 as everything stated in the article held true compared to the article I found, but it could have been more detailed.

One cited source that the article used was the graph that was attached to the article. This graph shows the comparison of mining and refining currently and in the future to other countries. The source is cited as S\&P Global Commodity Insights. I would give this a grade of 10 as the source can be easily read.

The article contains some examples of hyperbole. One example can be found in the following sentence: ``Prices doubled from record highs to end 2022 above \$75,000 per tonne, underlining the worsening shortage of the material crucial for electric car batteries.'' The use of ``record highs'' and ``worsening shortage'' is an exaggeration because it suggests that the situation is more extreme than it actually is. I would give the article a grade of 7 for avoiding exaggeration. Although the article contains some hyperbole, overall, it provided a factual account of the challenges the US is having.

The author presents some nuance in the article. One example is the following sentence: ``Even with all of this government backing, the challenges to boost supply remain formidable.'' This statement acknowledges that although the US government is taking steps to secure supplies of critical minerals, the task is still difficult and complex. I would give the article a grade of 8 for nuance. The author presents a nuanced analysis of the challenges the US is facing in the lithium inudstry. The article acknowledges the complex nature of the issue, including regulation and competition with other countries.

\end{document}