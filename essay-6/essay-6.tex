\documentclass{article}
\usepackage[utf8]{inputenc}

\title{Essay 6 - Conneciut Solar}
\author{Benny Chen}
\date{April 7, 2023}

\usepackage{color}
\usepackage{amsthm}
\usepackage{amssymb} 
\usepackage{amsmath}
\usepackage{listings}
\usepackage{xcolor}
\usepackage{listings}
\usepackage{graphicx}
\usepackage[hidelinks]{hyperref}

\definecolor{codegreen}{rgb}{0,0.6,0}
\definecolor{codegray}{rgb}{0.5,0.5,0.5}
\definecolor{codepurple}{rgb}{0.58,0,0.82}
\definecolor{backcolour}{rgb}{0.95,0.95,0.92}

\lstdefinestyle{mystyle}{
    backgroundcolor=\color{backcolour},   
    commentstyle=\color{codegreen},
    keywordstyle=\color{magenta},
    numberstyle=\tiny\color{codegray},
    stringstyle=\color{codepurple},
    basicstyle=\ttfamily\footnotesize,
    breakatwhitespace=false,         
    breaklines=true,                 
    captionpos=b,                    
    keepspaces=true,                 
    numbers=left,                    
    numbersep=5pt,                  
    showspaces=false,                
    showstringspaces=false,
    showtabs=false,                  
    tabsize=2
}

\lstset{style=mystyle}

\begin{document}

\maketitle


In this article, the author talks about how Conneciut is slowly moving to solar electrcity to decarbonize the secotr. One fact in the article was stating that solar power accounts for around 2.5\% of power in CT. This is a intresting fact of how much power that CT uses is just solar, and now is growing. I fact checked this with Sierra Club with also stated that ``Here in Connecticut, solar energy represents 2.53\% of electricity production which is enough to power 150,000 homes.''\footnote[1]{\url{https://www.connecticut.sierraclub.org/solarpanels}} This makes the statement true, and I would give the article a 10 for supporting its claims as there are many articles that also state these facts. In this article, there is also a cited source from the Coalition for Sensible Solar Regulation (CSSR)\footnote[2]{\url{https://solarforct.org}}. In the article it says that the CSSR estimates the cost of a electric bill if the wattage cap is lifted. I would give this article a 9 in having cited sources. There is a cited source and a link to the sources page but there arent many other sources in the article. The author does not use hyperbole in this article. They show the facts and data to support their argument without making extreme statements. I would give the author a 10 due to this. The author presents a clear argument with data and facts, without using hyperboles. They acknowledge the challenges and limitations of current regulations, but also offer solutions and opportunities to achieve their goal. The author does presents nuance by acknowledging the challenges and trade-offs involved in expanding solar power, such as the need to balance economic benefits with environmental conservation. For example, the author says ``That means a lot of ideal surface area goes unused, creating pressure to install larger utility-scale solar arrays in fields or woodlands. But those are the very places we need to conserve because they absorb carbon dioxide from the atmosphere, and because we need that land for farms, wildlife, clean water, our mental health, and for all life on the planet.'' I would give the author a 10 for this. The author shows nuance by showing the trade-offs of soalr power. They provide examples and evidence to support their argument while acknowledging the need to balance economic benefits with environmental conservation. 





\end{document}